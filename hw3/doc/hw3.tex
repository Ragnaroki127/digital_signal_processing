\documentclass[12pt]{ctexart}

\usepackage{amsmath}
\usepackage{amsfonts}
\usepackage[shortlabels]{enumitem}
\usepackage{authblk}
\usepackage{graphicx}
\usepackage{float}
\usepackage{titlesec}

\title{数字信号处理第三次作业}
\author{闫昊昕}
\date{\today}
\affil{2019210361}

\titleformat{\section}{\raggedright\large\bfseries}{第\,\thesection\,题}{1em}{}

\begin{document}
    \maketitle

    \pagestyle{empty}
    \section{}
    个人编写的n点FFT程序在fft.py中定义。该函数接受两个参数为函数输入:信号点序列$x_n$以及信号点数$N$。该函数要求信号点数严格为2的整数次幂。\par
    对一个输入16384点信号$f(n)=sin(2\pi\times 50n) + cos(2\pi\times 100n)$,个人编写的fft程序以及numpy包自带的fft程序得到的频谱图像如下:
    \begin{figure}[H]
        \centering
        \includegraphics[width=0.6\textwidth]{Figure_1.png}
        \caption{个人编写的基2FFT程序结果}
    \end{figure}
    \begin{figure}[H]
        \centering
        \includegraphics[width=0.6\textwidth]{Figure_2.png}
        \caption{Numpy自带的FFT程序结果}
    \end{figure} \par
    运行时间方面,个人编写的FFT程序可以在0.2秒左右的时间内完成对16384点信号的FFT运算,在运行速度上远远高于DFT算法,但和np.fft.fft函数相比运行速度仍然明显较慢,如下图所示:
    \begin{figure}[H]
        \centering
        \includegraphics[width=0.6\textwidth]{Figure_3.png}
        \caption{两种程序的运行时间比较}
    \end{figure}
    同时可以看到随着N的增加程序运行速度的变化约为线性,这与FFT算法的时间复杂度$O(n\log n)$相吻合。

    \section{}
    可以用两种方式实现函数的自相关,一种是根据定义实现序列的自相关,另一种方式是先对序列进行FFT得到其频谱,再做平方运算得到功率谱,最后做IFFT得到自相关函数的估计,本题中自相关函数的主要实现方式为定义法,功率谱法做验证使用。
    \begin{figure}[H]
        \centering
        \includegraphics[width=0.6\textwidth]{Figure_4.png}
        \caption{原信号的时域表示}\label{fig:original}
    \end{figure}
    \begin{figure}[H]
        \centering
        \includegraphics[width=0.6\textwidth]{Figure_5.png}
        \caption{根据定义实现的自相关函数}\label{fig:self}
    \end{figure}
    \begin{figure}[H]
        \centering
        \includegraphics[width=0.6\textwidth]{Figure_6.png}
        \caption{根据功率谱原理得到的自相关函数}\label{fig:val}
    \end{figure} \par
    图\ref{fig:original}为原信号的时域表示,图\ref{fig:self}为通过根据定义计算的自相关函数得到的自相关函数谱,图\ref{fig:val}为通过功率谱方法估计的自相关函数。可以看到原信号的自相关函数关于频谱中心对称,与理论相符合。

    \section{}
    定义原信号序列为:
    \begin{equation*}
        f(n) = 0.8\sin(2\pi\times 20n) + \cos(2\pi\times 50n) + 1.5\sin(2\pi\times 100n) + \epsilon
    \end{equation*} \par
    其中$\epsilon$为满足标准正太分布的随机噪声。信号长度为2s,采样点数为1024点,即采样频率为512Hz。该信号在时域的图像如下图所示:
    \begin{figure}[H]
        \centering
        \includegraphics[width=0.6\textwidth]{Figure_7.jpg}
        \caption{原信号的时域表示}
    \end{figure}
    对原信号进行补0操作,再求得自相关函数谱如下图:
    \begin{figure}[H]
        \centering
        \includegraphics[width=0.6\textwidth]{Figure_8.jpg}
        \caption{原信号的自相关谱}
    \end{figure} \par
    对该自相关谱做FFT,得到信号的功率谱图如下:
    \begin{figure}[H]
        \centering
        \includegraphics[width=0.6\textwidth]{Figure_9.jpg}
        \caption{自相关法得到的信号功率谱}
    \end{figure} \par
    可以清楚地看到功率谱在$f=20,50,100Hz$处出现峰值,这与原始信号的定义一致。
    接着对自相关谱进行时域加窗,窗长为1884点,即滤去两侧各100点的自相关谱信号,得到新的功率谱如下:
    \begin{figure}[H]
        \centering
        \includegraphics[width=0.6\textwidth]{Figure_10.jpg}
        \caption{加窗后的信号功率谱}
    \end{figure} \par
    信号的功率谱变得更为平滑,且出现了$sinc$函数特征,这是加矩形窗的结果。加窗前的功率谱方差为0.594,而加窗后的功率谱方差为0.287,功率谱方差有较大改善,但频谱分辨率也有降低。

    \section{}
    实现了对信号的Welch平滑改进,原信号长度为1024点,程序中对信号进行分段操作,每一段的长度为128点,两端信号的重合长度为64点,信号共被分为15段。得到的功率谱如下图:
    \begin{figure}[H]
        \centering
        \includegraphics[width=0.6\textwidth]{Figure_11.jpg}
        \caption{Welch平滑改进后的功率谱}
    \end{figure} \
    该频谱的方差为0.536,对比原始的功率谱方差有一定改进但同时功率谱的分辨率变得较差。

    \section{}
    假设某信号具有两个极为接近的频率峰值$f_1=100Hz$,$f_2=100.2Hz$,设该信号的表达式为:
    \begin{equation*}
        f(n)=0.8\sin(2\pi\times 100n)+1.2\cos(2\pi\times 100.2n)+\epsilon
    \end{equation*} \par
    该信号的时域图像如下:
    \begin{figure}[H]
        \centering
        \includegraphics[width=0.6\textwidth]{Figure_12.jpg}
        \caption{原信号的功率谱}
    \end{figure} \par
    根据周期图法得到信号的功率谱如下,可见两个频谱峰无法分辨:
    \begin{figure}[H]
        \centering
        \includegraphics[width=0.6\textwidth]{Figure_13.jpg}
        \caption{原信号的功率谱}
    \end{figure} \par
    现试着从以下三个方面对时域信号进行操作,探讨能否分辨两个频率峰:\par
    \begin{enumerate}
        \item 增加采样频率 \\
        对信号的采样频率增加16倍,即采样频率为8192Hz,得到新的功率谱在90到110Hz的图像如下:
        \begin{figure}[H]
            \centering
            \includegraphics[width=0.6\textwidth]{Figure_14.jpg}
            \caption{增加采样频率后的信号功率谱}
        \end{figure} \par
        可见两个峰值仍无法分辨,改种方法并不能提高频谱分辨率。
        \item 维持采样频率,增加信号实际长度 \\
        设定采样的信号长度为原信号的16倍,维持采样频率不变,得到的功率谱在90Hz到110Hz的图像如下:
        \begin{figure}[H]
            \centering
            \includegraphics[width=0.6\textwidth]{Figure_15.jpg}
            \caption{增加信号长度后的功率谱}
        \end{figure} \par
        这种方法成功地将两个谱峰分辨开,说明该方法可以有效地提高频谱分辨率。通过对两个峰值的解算可以得到两个峰值对应的频率正是100Hz和100.2Hz,说明改种方法可以准确地表征频率。
        \item 维持采样频率,用补0增加信号实际长度 \\
        将信号补0至原信号长度的16倍,得到新的功率谱在90Hz到110Hz的图像如下:
        \begin{figure}[H]
            \centering
            \includegraphics[width=0.6\textwidth]{Figure_16.jpg}
            \caption{时域补0后得到的的功率谱}
        \end{figure} \par
        该种方法得到的频谱图更为光滑,分辨率更低,因而无法分辨两个十分接近的谱峰。
    \end{enumerate}
\end{document}